\documentclass[12pt,a4paper]{resume}
%\documentclass[12pt,a4paper]{article}

\setname{Howard}{Kim}
%\title{something}{some city}{USA}
\address{23 Drake Lane}{West Lebanon, NH}{US}
\mobile{210.385.0589}
\email{howardfk @ gmail.com}
\homepage{http://www.dartmouth.edu/~howardkim}{www.dartmouth.edu/\textasciitilde howardkim} % 
% Sections Educatoin, work experince, skills, programing languges, ect
% sub sections, date ranges
%
%
\begin{document}
%make header/title name + contact info + websites

\makecvtitle

\section{EDUCATION}

\begin{school}{Dartmouth College}{2013 \textemdash \ \ 2015}{Masters of Science in Physics}{}{}{}
\end{school}

\begin{school}{University of Texas, San Antonio}{2010 \textemdash  \ \ 2013}{BS Physics}{GPA 3.5}{Highest Honors}{}
\end{school}

\begin{school}{Carnegie Mellon University}{2004 \textemdash 2008}{BS Economics}{}{Minor Physics}{}
\end{school}

\section{RESEARCH \& WORK}

\begin{work}{Ground-based Plasma Space Physics }{2013  \textemdash \ \ 2015}{Prof. James LaBelle, Dartmouth College }
\item{Create code with Python to calibrate data off reference signal}
\item{Model ray tracing to determine physical characteristics of hiss events}
\item{Use stistical and anlyitical methods to characterize hiss events}
\item{Use Matlab to create plots and quanitfy hiss events}
\item{Work on machinelarning to automaticly identify events of interest}
\end{work}

\begin{work}{Material Physics}{2011  \textemdash \ \ 2013}{Dr. ChongLin Chen, UTSA Physics}
\item{Use DC and RF sputtering to create Ni and Co samples on glass substrates}
\item{Create and troubleshoot LabVIEW code measuring resistance with AC/DC resistive bridges}
\item{\normalsize Acquire resistance data for samples over temperature, time, and environment}
\item{Design and build custom heater to measure thermoelectric properties across a gradient}
\end{work}

\begin{work}{Condensed Matter Modeling}{2011  \textemdash \ \ 2013}{Dr. Zlatko Koinov, UTSA Physics}
\item{Use DC and RF sputtering to create Ni and Co samples on glass substrates}
\item{Create and troubleshoot LabVIEW code measuring resistance with AC/DC resistive bridges}
\item{ Acquire resistance data for samples over temperature, time, and environment}
\item{Design and build custom heater to measure thermoelectric properties across a gradient}
\end{work}

\section*{Teaching}
\begin{work}{Teacher's Assistent}{2013  \textemdash \ \ 2014}{Dartmouth College}
\item{Graiding}
\item{Leading lab and comunicating}
\item{Helping with home work}
\end{work}

\begin{work}{Pennsylvania Governors School for the Sciences}{July 2007}{Physics TA/Counselor}
\item{Taught concepts in relativistic physics to high school seniors}
\item{Assisted project to build Wilberforce pendulum and documented findings}
\item{Evaluated students’ progress throughout the program}
\item{Encouraged intellectual thinking and social interaction between students}
\end{work}


\section{LANGUAGES \& SOFTWARE}
\WorkFont \small Python\ \ • \ \ numpy \ \ • \ \ scipy  \ \ • \ \ git\ \ • \ \   HTML5 \ \ • \ \ CSS3 \ \ • \ \ \LaTeX \ \ %• \ 
\\ MatLab\ \ • \ \ LabVIEW  • \ \ Mathamatic   \ \ • \ \  Photoshop


\section{HOBBIES}
WoodWorking  \hfill{} LetterPress \hfill{} Tae Kwon Do \hfill Web Design


\end{document}